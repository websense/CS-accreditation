% Matches ACS Accreditation 5.4 template section 3.2

\chapter{Institutional Context of ICT Programs}

\rubric{This chapter matches ACS Accreditation 5.4 template section 3.2 Institutional Context of ICT Programs}

\rubric{This section of the template identifies the evidence which would allow the Accreditation Panel to
assess the institutional context for ICT Education. Where the institution provides links to documents
or information, please ensure that the appropriate permissions to the relevant sections of the
institution’s systems are available to ACS panellists.}


\section{Institutional Commitment to ICT Education}

\subsection{Brief Strategic Statement of Institutional Support}
\rubric{Linking to an Institution mission statement and strategic plans may provide evidence of the
Institution’s long-term commitment to ICT. If necessary, a statement from the institution's president
or CEO may be needed.}


\subsection{ICT School Planning and Review}
\rubric{Existing documents embodying the School’s strategic directions for ICT education, industry
engagement, research and other professional activities are useful evidence.
Include the Development Plan if one was created during the School's self-analysis (see Accreditation
Manual Volume 1 Section 3).
Link to documents associated with the most recent School or curriculum review.}


\subsection{The ICT School – Structure and Institutional Context}
\rubric{In order for the panel to better understand the larger context of ICT programs, outline the
organisational structure of the School, including management roles and incumbents, and briefly how
it interacts with the institutional structure (Faculty and Institutional committees, etc).}

\subsection{Educational Location and Partnerships}
\rubric{List all campuses at which this program is offered. Include online as a separate campus.
If the program is offered by a third party in a partnership, please also supply contact details for the person in the partner organisation responsible for the institution’s ICT programs.}

\section{ICT Student Profile}
\rubric{While there are no specific ICT accreditation criteria beyond HESF Section 1.3, the panel needs an
understanding of the student profile or the New Zealand AQA expectation of reviews of courses and
program data. Link to an indication of the EFTSL in each program by campus and data concerning
student progression (admission and graduation data). The data should indicate student gender and
whether they are domestic or international.}


\section{Technological Resources for ICT Education}
\rubric{Identify specific ICT facilities, including laboratories, specialised technology and software in active use
for teaching and the level of student access to them.
Show how these facilities are related to current industry practice.
Indicate the technical support for these facilities and the training for staff.
Link to school and/or institutional policies regarding the use of technology in education.}

\section{Monitoring, Review and Improvement}

\rubric{No rubric provided - is this covered in Section 'ICT School Planning and Review' above?}

\section{Action from Previous Accreditation}
\rubric{Provide a response to the recommendations of the previous accreditation, when there was one.}

\rubric{TODO include actions taken}

\paragraph{Recommendations from ACS Accreditation October 2021}

It is recommended that the Department of Computer Science and Software Engineering:
\begin{enumerate}
\item As enrolment numbers increase, appoint additional continuing teaching staff to ensure that each academic staff member has sufficient time to incrementally improve their units and T\&R academic staff have sufficient time for research.

\begin{quote}
\color{blue}
{Action: 15 of the current CSSE staff, both T\&R and teaching focussed, have been appointed since the 2020 accreditation.}
\end{quote}

\item Consideration should be given to appointing a (possibly part-time) dedicated industry-based project/WIL/internship co-ordinator to reduce the administration workload for
capstone unit coordinators.

\begin{quote}
\color{blue}
{Action: The School Operations team has allocated support for our four capstone units.  The Work Integrated Learning units are supported by central UWA teams: McCusker Centre for Citizenship and UWA Student Employability and Career Development group.}
\end{quote}

\item Consider providing more opportunities for WIL and internships for academic credit.

\begin{quote}
\color{blue}
{Action: The option to undertake a WIL unit for credit has been added to the Masters of IT, Bachelor of Advanced Computer Science and Computer Science major.}
\end{quote}

\item Provide additional training and support to academic staff in the use of online collaboration
tools that will be required if online delivery continues or expands.
\item Provide training for Program Chairs in SFIA to assist them in future program development
and modification of existing programs.
\item In programs where the Mathematical Foundations unit is compulsory, consider replacing
the unit with a mathematics unit that is more relevant to the ICT units within the programs.
\item As part of program design, review and update unit and program outcomes to ensure that
comprehensive alignment (guided by SFIA \& CBOK).

\begin{quote}
\color{blue}
{Action: All outcomes were carefully reviewed in 2021, and are regularly reviewed in UWA's annual curriculum review process.}
\end{quote}

\item Ensure that the loop is closed on student feedback from student surveys and staff/student meetings.

\begin{quote}
\color{blue}
{Action: TODO csfeedback page, notes of SSL.}
\end{quote}


\item Review pre-requisites for advanced units and capstone units and ensure that they are
enforced so that students taking the unit have appropriate skills.

\begin{quote}
\color{blue}
{Action: Updated outcomes TODO state them.}
\end{quote}


\item Consider including content on system deployment in all programs

\begin{quote}
\color{blue}
{Action: TODO missing semester TODO ACS PPP.}
\end{quote}

\item Consider increasing the number of guest lectures delivered by industry partners.

\begin{quote}
\color{blue}
{Action: DONE.  Evidence?.}
\end{quote}

\item Consider including the action taken in response to student feedback in SURF surveys in
subsequent unit outlines to demonstrate to students that their feedback is considered and
acted upon.

\begin{quote}
\color{blue}
{Action: TODO some examples eg \hyperlink{https://teaching.csse.uwa.edu.au/units/CITS2002/feedback/}{Recent CITS2002 Student Feedback}
}
\end{quote}


\item Encourage staff to attend student club meetings such as the Data Science Club industry nights.

\begin{quote}
\color{blue}
{Action: Done}
\end{quote}

\item Consider facilitating students to sit for industry certification examinations where appropriate.

\begin{quote}
{\color{blue}
{Action: TODO any examples?  AWS was tried? Clubs?}
}
\end{quote}

\item Encourage staff to actively participate in the Discord server discussions.
\item Review the tools and platforms used in labs to ensure that they are reasonably up to date.

\begin{quote}
{\color{blue}
{Action: Not done. Distinguish between official help forum and student discussion groups}
}
\end{quote}

\item Consider requiring rubrics to be developed and used for all assessments to ensure consistency of marking.
\begin{quote}
\color{blue}
{Action: Done as per UWA assessment policy.  
Staff supported by CSSE workshop on assessment and rubrics.}
\end{quote}

\end{enumerate}

\section{Generative AI}
\rubric{Provide a link to the institution’s policy on GenAI and/or to the ICT School’s, if one exists}
