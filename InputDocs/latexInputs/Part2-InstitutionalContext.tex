% Matches ACS Accreditation 5.4 template section 3.2

\chapter{Institutional Context of ICT Programs}

\rubric{This chapter matches ACS Accreditation 5.4 template section 3.2 Institutional Context of ICT Programs.
This section of the template identifies the evidence which would allow the Accreditation Panel to
assess the institutional context for ICT Education. Where the institution provides links to documents
or information, please ensure that the appropriate permissions to the relevant sections of the institution’s systems are available to ACS panel.}


\section{Institutional Commitment to ICT Education}

\rubric{Linking to an Institution mission statement and strategic plans may provide evidence of the
Institution’s long-term commitment to ICT. If necessary, a statement from the institution's president or CEO may be needed.}

\subsection{Brief Strategic Statement of Institutional Support}



\subsection{ICT School Planning and Review}

\rubric{Existing documents embodying the School’s strategic directions for ICT education, industry
engagement, research and other professional activities are useful evidence.
Include the Development Plan if one was created during the School's self-analysis (see Accreditation Manual Volume 1 Section 3).}

 
\subsection{The ICT School – Structure and Institutional Context}
\rubric{In order for the panel to better understand the larger context of ICT programs, outline the organisational structure of the School, including management roles and incumbents, and briefly how it interacts with the institutional structure (Faculty and Institutional committees, etc).}


\subsection{Educational Location and Partnerships}
\rubric{List all campuses at which this program is offered. Include online as a separate campus.
If the program is offered by a third party in a partnership, please also supply contact details for the person in the partner organisation responsible for the institution’s ICT programs.}


\section{ICT Student Profile}
\rubric{While there are no specific ICT accreditation criteria beyond HESF Section 1.3, the panel needs an
understanding of the student profile or the New Zealand AQA expectation of reviews of courses and
program data. Link to an indication of the EFTSL in each program by campus and data concerning
student progression (admission and graduation data). The data should indicate student gender and
whether they are domestic or international.
}


Directory Link: \AccreditationSharedDrive 




\section{Technological Resources for ICT Education}
\rubric{Identify specific ICT facilities, including laboratories, specialised technology and software in active use
for teaching and the level of student access to them.
Show how these facilities are related to current industry practice.
Indicate the technical support for these facilities and the training for staff.
Link to school and/or institutional policies regarding the use of technology in education.

``In addition to general infrastructure there will be specific ICT hardware and software technology to fully support the achievement of the specified learning outcomes for each program.  The technology will be reasonably representative of contemporary ICT practice"}




\section{Monitoring, Review and Improvement}

\rubric{Criteria 3.1.4: 
Accredited ICT programs will be guided by a formally constituted ICT Industry Advisory Board or
mechanism involving industry stakeholders, particularly local employers. This body is expected to
operate at the strategic level in monitoring and analysing ICT industry needs and trends and ensuring
that they influence program design and subject teaching. It will monitor the achievement of program
objectives and graduate capability targets.
}

%\TODO{Add documents associated with the most recent curriculum reviews (MIT, CS) (as allowed) to the Accreditation shared drive.} 

%\paragraph{Industry Advisory Panel}

%\bigskip


\input{Tables/IAPtable}

Click on an IAP member's name for their LinkedIn pages




\section{Action from Previous Accreditation}
\rubric{Provide a response to the recommendations of the previous accreditation, when there was one.}

\subsubsection*{\em Recommendations of ACS Accreditation Report October 2021 with Actions Taken}

\begin{longtable}{| p{1\textwidth} |} %multipage table
\hline
\cellcolor{colorLightBlue} 
(1) Recommendation
\\ \hline
\\
{\bf Action:} Action Taken

\\ \hline
\cellcolor{colorLightBlue} 
(2) Recommendation
\\ \hline
\\
{\bf Action:} Action Taken

\\ \hline
\end{longtable}

\section{Generative AI}
\rubric{Provide a link to the institution’s policy on GenAI and/or to the ICT School’s, if one exists. d) Generative AI
ACS believes all students should have opportunities to learn about GenAI and how to use
it responsibly. Institutions should have a policy on GenAI. However, the ACS will not use
either the presence or absence of any such policies as a criterion in the accreditation of
the program. Regardless of an institution's policy, in the light of the ease of access to Gen
AI, it is expected that institutions will make appropriate adjustments to relevant
assessments in order to maintain academic integrity.}


%==== END OF PART 2 ====