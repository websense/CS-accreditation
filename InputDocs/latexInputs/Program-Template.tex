%THISPROG to be replaced with COURSE eg MJD-CMPSC througout
% TODO THISPROG as section header
% Chapter with Sectionss
\label{sec:THISPROG}

% Critierion A : program design
\section{Criterion A. Program Design}
\rubric{Provide a link to the institution-approved document(s) that specify the programs approved objectives and learning outcomes (including the institutions graduate outcomes), entry requirements and structure. This will usually be the program website but may be an online handbook.}

\input{./Tables/criterionA-THISPROG.tex}

% Critierion B : SFIA
\section{Criterion B. Professional ICT Role and Skills}
\rubric{For each named major or specialisation in the program (or for the program itself where there is no major), identify the primary ICT professional role that a graduate is equipped to perform on graduation. While aspirational roles (such as CIO, project manager) have a purpose in indicating a career path for graduates, the professional role identified here will be one to which the graduate is immediately suited. If necessary, one further role may be listed.}

\input{./Tables/criterionB-THISPROG.tex}

% %CriterionC: Coverage of ICT Knowledge
\newpage
\section{Criterion C. Coverage of ICT Knowledge}
\rubric{To demonstrate breadth of ICT knowledge, use a grid (see sample below) to associate major topics from the CBoK with the most significant mandatory subjects that assess that knowledge. Identify a maximum of 3 subjects in Professional and Core columns. No column may be empty. Level 1 is Introductory know-that Bloom 1 and 2. Level 2 means Intermediate know-how Bloom 3. Level 3 means Mature know-why Bloom 4 and 5}

\begin{center}
\begin{tabular}{|p{2cm}| *{14}{p{0.28cm}|}}
\hline
\footnotesize{\rotatebox{90}{CBoK Knowledge Areas}} & 
\footnotesize\cellcolor{paleOrange}{\rotatebox{90}{ICT Ethics}} & 
\footnotesize\cellcolor{paleOrange}{\rotatebox{90}{Impacts of ICT} }& 
\footnotesize\cellcolor{paleOrange}{\rotatebox{90}{Working Individually and Teamwork} }& 
\footnotesize\cellcolor{paleOrange}{\rotatebox{90}{Professional Communication}} & 
\footnotesize\cellcolor{paleOrange}{\rotatebox{90}{Professional Practitioner}} & 
\footnotesize\cellcolor{paleGreen}{\rotatebox{90}{ICT Fundamentals} }& 
\footnotesize\cellcolor{paleGreen}{\rotatebox{90}{ICT Infrastructure} }& 
\footnotesize\cellcolor{paleGreen}{\rotatebox{90}{Information and Data Science and Engineering} }& 
\footnotesize\cellcolor{paleGreen}{\rotatebox{90}{Computational Science and Engineering} }& 
\footnotesize\cellcolor{paleGreen}{\rotatebox{90}{Application Systems} }& 
\footnotesize\cellcolor{paleGreen}{\rotatebox{90}{Cyber Security}} & 
\footnotesize\cellcolor{paleGreen}{\rotatebox{90}{ICT Project Management}} & 
\footnotesize\cellcolor{paleGreen}{\rotatebox{90}{ICT management and governance}} & 
\footnotesize\cellcolor{paleBlue}{\rotatebox{90}{In-depth ICT Knowledge}} \\ \hline

\footnotesize{Mandatory Units} & 
        \multicolumn{5}{|c|}{\footnotesize\cellcolor{paleOrange}{\bf Professional}} & 
        \multicolumn{8}{|c|}{\footnotesize\cellcolor{paleGreen}{\bf Core}} & 
        \multicolumn{1}{|c|}{\footnotesize\cellcolor{paleBlue}{}} \\
\hline
%insert table body rows and footer (only), no head
\input{./Tables/criterionC-THISPROG.tex}
\end{center}



% Criterion D. Advanced ICT Knowledge Addressing Complex Computing
\newpage
\section{Criterion D. Advanced ICT Knowledge Addressing Complex Computing}
\rubric{Review the requirements for advanced ICT knowledge to address complex computing in the
Accreditation Manual, Volume 2, Criterion D.\\
In the table below:\\
Identify subjects that are assessed at an advanced level that are targeted specifically at the
professional role identified for this program (exclude the advanced subjects used in
Criterion E).
Identify the assessment item(s) that assess ICT knowledge at an advanced level e.g., at least
Blooms level 4). Note that quizzes, exams, practicals, etc do not count. The assessment
item must involve the solution of a complex problem with few directions as to the
solution approach and would typically have a weighting of at least 25\% of the subject
total
Then explain in no more than 50 words which Seoul Accord criteria of complex computing are
addressed by each such assessment item (see Criterion D in Volume 2 of the
Accreditation Manual).}

\input{./Tables/criterionD-THISPROG.tex}

% Criterion E. Integrated and Applied ICT Knowledge and Skills
\newpage
\section{Criterion E. Integrated and Applied ICT Knowledge and Skills}
\rubric{Review the requirements for integrated and applied ICT knowledge in the Accreditation Manual, Volume 2, Criterion E. Identify the advanced subject(s), often a capstone, that provide and assess the integration of knowledge and skills specifically targeted at the professional role identified for this program.}

\input{./Tables/criterionE-THISPROG.tex}

%Criterion F. Preparation for Professional ICT Practice
\section{Criterion F. Preparation for Professional ICT Practice}
\rubric{Show, {\bf in no more than 100 words}, how the program develops a well-rounded professional with respect to the attributes listed in the Accreditation Manual, Volume 2, Criterion F. This particularly refers to opportunities for WIL, internships, etc.}

\input{./Tables/criterionF-THISPROG.tex}




