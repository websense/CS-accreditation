%Template (ground truth version) for generating ACS Programs details for each program
%THISPROG is replaced with COURSE eg MJD-CMPSC througout


\label{sec:THISPROG}

%Intro

\subsection*{Program Details and Personnel}

\rubric{Prog code, Award title on testamur, ditto on transcript, campuses where offered, Program coordinator}

\input{./Tables/criterionA-THISPROG.tex}

%\subsection*{Access to Documents and Teaching Materials}
% All covered in the links above
%\rubric{Provide a link to the institution-approved document(s) that specify the program and each subject including, or as well as, all assessment items and how they are matched to subject learning outcomes. For most institutions, this will be in the Learning Management System.\\
%Provide read-only, auditor level access to the Learning Management System. This access is used to review and assess the teaching and learning as it applies to accreditation criteria. The access should cover all subjects being taught so is usually to the last two semesters.  Access should not be provided for the current semester.}
%
%\begin{tabular}{l l}
%\toprule
%Category & Source \\
%\midrule
%Program and each subject & See Handbook link to the program above \\
%Assessment items and how they are matched to subject learning outcomes & \UWALMS Unit Outline \\
%
%Objectives, Learning Outcomes, Entry Requirements & Program Web Page, Handbook \\ \hline
%Program Structure & Handbook (Study Plans), Caidi Report \\ hline
%Components, Structure & Handbook (Study Plans) \\ \hline
%\bottomrule
%\end{tabular}


%\begin{itemize}
%\item For a  listing of the {\em program and each subject} see the \textcolor{blue}{Program Specification, Units, and Study Plans} link in the summary table above.
%
%\item For mapping of per-unit assessment items to learning outcomes see the {\em Unit Outline} 
%menu item in the \UWALMS for each subject.  The accreditation team has access to the \UWALMS  for all semester 2 2024 and semester 1 2025 subjects.
%%e.g. \url{https://alt-5ddb108fe0c42.blackboard.com/bbcswebdav/institution/Unit_Outlines_2025/CITS5206_SEM-1_2025/CITS5206_SEM-1_2025_UnitOutline.html}.  Each unit includes a 
%\end{itemize}
\newpage
\section{Criterion A. Program Design}

%% TODO REUTRN HERE DECIDE HOW TO SUMMARISE LINKS TO INFO REQUESTED - TABLE OR TEXT?
%\begin{tabular}{l l}
%\toprule
%Category & Source \\
%\midrule
%Objectives, Learning Outcomes, Entry Requirements & Program Web Page, Handbook \\ \hline
%Program Structure & Handbook (Study Plans), Caidi Report \\ hline
%Components, Structure & Handbook (Study Plans) \\ \hline
%\bottomrule
%\end{tabular}


\subsection*{Program Objectives and Outcomes}
\rubric{Provide a link to the institution-approved document(s) that specify the programs approved 1) objectives and  2) learning outcomes (including the institutions graduate outcomes), 3) entry requirements and 4) structure. This will usually be the program website but may be an online handbook.\\
TODO learning outcomes for majors are listed in the handbook eg \url{https://handbooks.uwa.edu.au/majordetails?code=MJD-CMPSC} but not for masters eg \url{https://handbooks.uwa.edu.au/coursedetails?code=62510} }

\begin{itemize}
\item  For details of {\em program objectives, learning outcomes, entry requirements and structure} see the
Program Handbook and Program Web Page links in the Table above.  
%\textcolor{blue}{Program Specification, Units, and Study Plans} link in the summary table above.
%The program page also gives links to the handbook entry for each unit in the program.

%%not yet since not requested
%\item For mapping of unit outcomes to program outcomes see 
%\ACSTablesHTML ~ and \ACSTablesExcel.

\item Further details about the program including changes approved for 2026 can be found in \CaidiReports.
\end{itemize}

\subsection*{Program Components and Structure}
\rubric{Link to info that identifies mandatory ICT subjects, elective ICT subjects and non-ICT subjects, prereq knowledge links.\\
Provide a list of any mandatory ICT subjects that are sourced from external providers through articulation agreements.}


%See above Link handbook description (handbook includes colour coded study plans)
\begin{itemize}
\item  For a  listing of the {\em mandatory and elective ICT subjects and non-ICT subjects and prerequisites} see the
Program Handbook and Program Web Page links in the Table above.
Colour-coded Study Plan documents are provided in the Program Handbook.

\item For mapping of assessment items to learning outcomes in each subject see the {\em Unit Outline} 
menu item in the \UWALMS for each subject.  The accreditation team has access to the \UWALMS  for all mandatory subjects for semester 2 2024 and semester 1 2025.

\item  See Section~\ref{Pathways} of this document for a list of mandatory ICT subjects that may be {\em sourced from external providers} through articulation agreements.
\end{itemize}

\subsection*{Justification of Program Design}
\rubric{
There will be a justification for the program founded on the needs of stakeholders, including
employers, graduates and the student intake; international curricula relevant to the program’s field
of education and practice; and comparisons with programs of a similar nature available nationally or
internationally.

Show coherence of title, objectives, components and structure.   Explaining:
\begin{enumerate}
\item is related to needs of future stakeholders (ICT industry and community)
\item responds to international curricula
\item compares with programs of a similar nature nationally or internationally
\item embodies good program design practices from current academic literature
\item embodies trends in professional practice
\end{enumerate}
Provide a link to docs that justifies the program (recent prog review).  

TODO Links to further docs about the program that may assist the panel to evaluate its fit with its professional environment.  ACM, IEEE curricula and SWEBOK, BABOK, CYBOK etc.
}

% TODO break this into paragraphs and tidy up all the links in sep template sections

\input{./Tables/Justification-THISPROG.tex}

%Links to recent program reviews are provided in the \AccreditationSharedDrive this is covered in Part 1

% Critierion B : SFIA
\newpage
\section{Criterion B. Professional ICT Role and Skills}
\rubric{For each named major or specialisation in the program (or for the program itself where there is no major), identify the primary ICT professional role that a graduate is equipped to perform on graduation. While aspirational roles (such as CIO, project manager) have a purpose in indicating a career path for graduates, the professional role identified here will be one to which the graduate is immediately suited. If necessary, one further role may be listed.

The SFIA skills required to fulfil the ICT professional role (or roles) will be specified. Normally there
would be {\bf 2 primary SFIA skill areas in a particular professional role}. Subjects in the program which
{\bf assess} these skills will be {\bf mapped} to demonstrate that graduates achieve the underlying skills at SFIA
level 3 (or above).

Table of SFIA Skill Code, SFIA level	Subject Code, Title plus a separate mapping to justify (as above)
}



\input{./Tables/criterionB-THISPROG.tex}

\bigskip
For a mapping demonstrating how graduates achieve the underlying skills at SFIA
level 3 (or above) and how these skills are assessed in the units see  
\ACSTablesHTML.



% %CriterionC: Coverage of ICT Knowledge
\newpage
\section{Criterion C. Coverage of ICT Knowledge}
\rubric{To demonstrate breadth of ICT knowledge, use a grid (see sample below) to associate major topics from the CBoK with the most significant mandatory subjects that assess that knowledge. Identify a maximum of 3 subjects in Professional and Core columns. No column may be empty. Level 1 is Introductory know-that Bloom 1 and 2. Level 2 means Intermediate know-how Bloom 3. Level 3 means Mature know-why Bloom 4 and 5}

\begin{center}
\begin{tabular}{|p{2.8cm}| *{14}{p{0.26cm}|}}
\hline
\footnotesize{\rotatebox{90}{CBoK Knowledge Areas}} & 
\footnotesize\cellcolor{paleOrange}{\rotatebox{90}{Professional ICT Ethics}} & 
\footnotesize\cellcolor{paleOrange}{\rotatebox{90}{Impacts of ICT} }& 
\footnotesize\cellcolor{paleOrange}{\rotatebox{90}{Working Individually and Teamwork} }& 
\footnotesize\cellcolor{paleOrange}{\rotatebox{90}{Professional Communication}} & 
\footnotesize\cellcolor{paleOrange}{\rotatebox{90}{Professional ICT Practitioner}} & 
\footnotesize\cellcolor{paleGreen}{\rotatebox{90}{ICT Fundamentals} }& 
\footnotesize\cellcolor{paleGreen}{\rotatebox{90}{ICT Infrastructure} }& 
\footnotesize\cellcolor{paleGreen}{\rotatebox{90}{Information and Data Science and Engineering} }& 
\footnotesize\cellcolor{paleGreen}{\rotatebox{90}{Computational Science and Engineering} }& 
\footnotesize\cellcolor{paleGreen}{\rotatebox{90}{Application Systems} }& 
\footnotesize\cellcolor{paleGreen}{\rotatebox{90}{Cyber Security}} & 
\footnotesize\cellcolor{paleGreen}{\rotatebox{90}{ICT Projects}} & 
\footnotesize\cellcolor{paleGreen}{\rotatebox{90}{ICT Management and Governance}} & 
\footnotesize\cellcolor{paleBlue}{\rotatebox{90}{In-depth ICT Knowledge}} \\ \hline

\footnotesize{Mandatory Units} & 
        \multicolumn{5}{|c|}{\footnotesize\cellcolor{paleOrange}{\bf Professional}} & 
        \multicolumn{8}{|c|}{\footnotesize\cellcolor{paleGreen}{\bf Core}} & 
        \multicolumn{1}{|c|}{\footnotesize\cellcolor{paleBlue}{}} \\
\hline
%insert table body rows and footer (only), no head
\input{./Tables/criterionC-THISPROG.tex}

%\end{tabular} is included in the input .tex file
%\end{center} is included in the input .tex file then footnote

For mappings demonstrating how the most significant mandatory units assess these ICT Knowledge areas
see  \ACSTablesHTML.
%{\color{red} TODO add link to html \url{teaching.case.uwa.edu.au/acs-accreditation/program-mjd-aridm.html}}.


% Criterion D. Advanced ICT Knowledge Addressing Complex Computing
\newpage
\section[Criterion D. Advanced ICT Knowledge Addressing Complex Computing]
{Criterion D. Advanced ICT Knowledge\\ Addressing Complex Computing} %force sensible line break

\rubric{Review the requirements for advanced ICT knowledge to address complex computing in the
Accreditation Manual, Volume 2, Criterion D.\\
In the table below:\\
Identify subjects that are assessed at an advanced level that are targeted specifically at the
professional role identified for this program ({\bf exclude} the advanced subjects used in
Criterion E).
Identify the assessment item(s) that assess ICT knowledge at an advanced level e.g., at least
Blooms level 4). Note that quizzes, exams, practicals, etc do not count. The assessment
item must involve the solution of a complex problem with few directions as to the
solution approach and would typically have a weighting of at least 25\% of the subject
total
Then explain in no more than 50 words which Seoul Accord criteria of complex computing are
addressed by each such assessment item (see Criterion D in Volume 2 Criteria of the
Accreditation Manual).
A program will have at least 4 advanced subjects, 
counting no more than 2 breadth-based subjects (E - but see above exclude capstone from list),
at least 2 depth-based subject of ICT discipline-specific knowledge directly related to the professional role and SFIA for the program (so not project management or ethics units).
}

\input{./Tables/criterionD-THISPROG.tex}

\bigskip
\noindent{
Numbers in the right column refer to  {\bf Complex Computing Criteria}:
(1) involves wide-ranging or conflicting technical, computing, and other issues;
(2) has no obvious solution, and requires conceptual thinking and innovative analysis to
formulate suitable abstract models;
(3) a solution requires the use of in-depth computing or domain knowledge and an analytical
approach that is based on well-founded principles;
(4) involves infrequently encountered issues;
(5) is outside problems encompassed by standards and standard practice for professional
computing;
(6) involves diverse groups of stakeholders with widely varying needs;
(7) has significant consequences in a range of contexts;
(8) is a high-level problem possibly including many component parts or sub-problems;
(9) identification of a requirement or the cause of a problem is ill defined or unknown.
(Seoul Accord, Section D) [Source: ACS Accreditation 5.4 Volume 2 Criteria August 2024]}

%{Note the capstone unit for this program (see E) is also an advanced, breadth-based unit, but is excluded from this table as per the ACS Application Template instructions} Leave it in for now

% Criterion E. Integrated and Applied ICT Knowledge and Skills
\newpage
\section[Criterion E. Integrated and Applied ICT Knowledge and Skills]{Criterion E. Integrated and Applied ICT\\ Knowledge and Skills} %force sensible line break
\rubric{Review the requirements for integrated and applied ICT knowledge in the Accreditation Manual, Volume 2, Criterion E. Identify the advanced subject(s), often a capstone, that provide and assess the integration of knowledge and skills specifically targeted at the professional role identified for this program.}

\input{./Tables/criterionE-THISPROG.tex}


%Criterion F. Preparation for Professional ICT Practice
\bigskip
\section{Criterion F. Preparation for Professional ICT Practice}
\rubric{Show, {\bf in no more than 100 words}, how the program develops a well-rounded professional with respect to the attributes listed in the Accreditation Manual, Volume 2, Criterion F. This particularly refers to opportunities for WIL, internships, etc.  }

\input{./Tables/criterionF-THISPROG.tex}




