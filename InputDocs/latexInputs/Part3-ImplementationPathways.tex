% ACS Accreditation 5.4 template section 3.3.2 Program Implementation Pathways

\rubric{In this section of the template, link to any information needed to explain and justify any relevant
aspects of the program implementation pathways referred to in
Section 3.2.2 of the Accreditation Manual Volume 2 Accreditation Criteria.
This typically refers to undergraduate degrees which may enrol students in higher years based on
students’ previous studies in partner feeder programs.}


\section{Educational Locations and Partnerships}

\rubric{In the following, Satellite Campus refers to a campus of the institution located separately from the
main campus but with the institution’s name and with academic staff appointed by and employed by
the institution whose programs are being accredited. Partner Campus refers to a third party
organisation running programs for the institution. Typically, academic staff are appointed and
employed by the partner but approved by the institution whose programs are being accredited. 
In addition, for each program offered at each additional location, provide a list of staff who teach
into the program (in the same format as for the main campus) and provide a link to a short CV for
each of these staff members.  TBA is this required for our articulation programs?
}

\rubric{
No programs are offered at a Satellite Campus.
Several articulation agreements exist with Partner universities.
The agreements specify where credit is given for prior learning when students transfer to UWA.

For each additional location (other than the main campus) where a program is offered, provide
information on the following aspects.
}

%• For a Satellite Campus: name of location
\begin{quote}
For a Partner Campus:
\begin{itemize}
\item Name of location(s) and partnership organisation
\item Nature of relationship with main institution – include relevant dates for collaboration
\item Model of program delivery e.g., on shore/off shore combination (3+1), fully off shore (4+0), etc.
\item Is the partner organisation an accredited educational institution? E.g., TEQSA, other government body.
\item QA roles and processes that are in place for the program delivery
\item Roles and processes for employing local staff, including types of employment contracts
\begin{itemize}
\item Program operational handbook which determines each party’s roles and responsibilities
\item Roles and processes for assigning teaching staff to subjects, included (if relevant) subject
coordinator, moderator, and lecturing staff (lecturers and tutors)
\item Available subject offerings and program requirements: are all subjects (including electives) offered
at each location, are there additional/different program completion requirements?
\item Timing of subject offerings and assessment tasks: are they offered at the same time, at different
times, or a mix of the two?
\item Roles and processes for setting assessment in the course including assignments and exams
\item  Roles and processes for moderating assessed items and results
\item  Roles and processes for student support for subjects and more generally
\item Roles that staff from each location have in curriculum development and review
\item  If this information has not been provided elsewhere in the submission, number of students enrolled
(commencing and continuing) and number of completions in each program at each location
\end{itemize}
\end{itemize}
\end{quote}


